\section{Discussion}

From the experiments,
we can observe that
\begin{itemize}
    \item In the Frobenius norm setting,
        most of the time was spent on the second phase,
        \ie solving the smaller sized problem.
        More specifically,
        Almost all time was used in the multiplication $AQ$.
    \item To accelerate this multiplication,
        we tried multiple methods.
        One of the interesting fact is that,
        by converting matrix $A$ from sparse to full,
        the multiplication will be much faster
        (from 50s to 5s).
        This might due to factors like cache or memory alignment.
        It would be good if we can use sketching
        that keep operator norm of product of two matrices
        to accelerate this multiplication.
\end{itemize}
