\section{Sketching}

\subsection{Frobenius norm}
For simplicity, we use sparse embedding, \ie the count sketch
to find a subspace that may contain a good low-rank approximation to $A$.
The general steps in \textit{sketch\_frob()} function are:
\begin{enumerate}
    \item Input: matrix $A\in \real^{138493\times 26744}$,
        rank $k=128$,
        $t\in\{256, 512, 1024\}$ which is the the sketch dimension.
    \item Construct count sketch matrix $S$.
    \item Do qr decomposition $Q, R = \text{qr}((SA)^T)$.
    \item Do svd $U, S, V = \text{svd}(AQ)$.
    \item Truncate to $k$-dimensions $U_k$, $S_k$, and $V_k$.
    \item Output: $\tilde{A}_k = U_k S_k V_k^T Q^T$.
\end{enumerate}

We divide the process into two phases and count the running time independently.
The first phase is sketching and
the second phase includes all the remaining steps
which is to solve a reduced version of the problem.
Results are shown in Table~\ref{tab:frob}.

\begin{table}[htb]
  \setlength{\tabcolsep}{2.6pt}
  \caption{The performances of different sketching techniques.
  $k$ is the rank used, $t$ is the dimension for sketching.
  time-$i$ is the average time spent on phase~$i$.
  Losses are measured by Frobenius norms.
    }
  \label{tab:frob}
  \centering
  {\small
  \begin{tabular}{lllllllllllll}
    \toprule
    algorithm & repeat & $k$ & $t$ & time1\&2 & time & time ratio & min loss & max loss & mean loss & std loss & loss ratio \\
    \midrule
    Baseline & 1 & 128 & - & - & 111.4 & 1 & 11622.2 & 11622.2 & 11622.2 & - & 1 \\
    Sketching & 100 & 128 & 256 & 2.7/13.8 & 16.5 & 0.148 & 12664.1 & 12693.7 & 12677.0 & 6.2537 & 1.089 \\
    Sketching & 100 & 128 & 512 & 2.8/23.8 & 26.6 & 0.239 & 12282.5 & 12298.3 & 12290.0 & 3.1376 & 1.056 \\
    Sketching & 100 & 128 & 1024 & 2.8/49.8 & 52.6 & 0.472 & 11971.6 & 11979.0 & 11974.7 & 1.314 & 1.030 \\
    \bottomrule
  \end{tabular}
  }
\end{table}

\subsection{Operator norm}
The operator norm problem was tackled by the subspace power method.
We implemented it in the \textit{sub\_power()} function.
The general steps go like this,
\begin{enumerate}
    \item Input: matrix $A\in \real^{138493\times 26744}$,
    rank $k=128$, power $q\in\{10,20,40\}$.
    \item Generate random Gaussian matrix $G\in\real^{26744\times k}$.
    \item $Y=A*G$.
    \item repeat $Y = A^T Y, Y=AY$, for $q$ times.\
    \item Do qr factorization $Z, \_ = \text{qr}(Y)$.
    \item Output: $\tilde{A}_k = Z Z^T A$.
\end{enumerate}
